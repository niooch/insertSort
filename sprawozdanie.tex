\documentclass{article}
\usepackage{graphicx}
\usepackage{amsmath}
\usepackage{polski}
\usepackage[utf8]{inputenc}
\usepackage{hyperref}

\hypersetup{
    colorlinks=true,
    linkcolor=blue,
    filecolor=magenta,
    urlcolor=cyan,
}

\title{Symulowanie Procesów Losowych - Algorytm Insert Sort}
\author{Jakub Kogut}
\date{}

\begin{document}

\maketitle

\section{Wstęp}
Sprawodzanie do zadania domowego 3, zadanie 2. -- \textit{Sortowanie przez wstawianie losowych danych}.
\section{Opis Zadania}
Zadanie polegało na zaimplementowaniu algorytmu sortowania przez wstawianie oraz przeprowadzeniu symulacji dla różnych wartości $n$. \newline
Celem zadania było wyznaczenie ilości przestawień oraz porównań w poszczególnych powtórzeniach działania alorytmu.
\section{Metodologia}
Podana była ustalona wartość $k=50$ powtórzeń eksperymentu dla każdego $n$. Wartość $n$ miała pochodzić ze zbioru $\{k \times 10^2: k \in \{1, ..., 10^2\}\}$. Dla każdego $n$ wyznaczano średnią z $k$ powtórzeń eksperymentu. Następnie wyznaczano odpowiednio wartość cmp(n) oraz s(n) dla każdego $n$.
\section{Wnioski}
Na podstawie przeprowadzonych symulacji można zauważyć, że obydwie wartości $cmp(n)$ oraz $s(n)$ rosną kwadratowo. Koncentracja poszczególnych wyników wokół średniej wartości jest wysoka.
\subsection{Porównania $cmp(n)$}
Na \hyperref[fig:cmp]{wykresie} można zauważyć, że wartość $cmp(n)$ rośnie kwadratowo. Wartość $cmp(n)$ jest zależna od ilości porównań w algorytmie. Asymptotyke jasno można określić jako $O(n^2)$ przyglądając się \hyperref[fig:cmpOvern2]{wykresowi.}
\subsection{Przestawienia $s(n)$}
Na \hyperref[fig:s]{wykresie} można zauważyć, że wartość $s(n)$ rośnie kwadratowo. Wartość $s(n)$ jest zależna od ilości przestawień w algorytmie. Asymptotyke jasno można określić jako $O(n^2)$ przyglądając się \hyperref[fig:Ln1]{wykresowi.}
\section{Podsumowanie}

\section{Wykresy}
\begin{figure}[ht]
    \centering
    \includegraphics[width=0.8\textwidth]{wykresy/porownania.png}
    \caption{Wykres watosci $cmp(n)$}
    \label{fig:cmp}
\end{figure}
\begin{figure}[ht]
    \centering
    \includegraphics[width=0.8\textwidth]{wykresy/przestawienia.png}
    \caption{Wykres watosci $s(n)$}
    \label{fig:s}
\end{figure}
\begin{figure}[ht]
    \centering
    \includegraphics[width=0.8\textwidth]{wykresy/przestawieniaOVERn.png}
    \caption{Wykres wartości $\frac{s(n)}{n}$}
    \label{fig:sOvern}
\end{figure}
\begin{figure}[ht]
    \centering
    \includegraphics[width=0.8\textwidth]{wykresy/przestawieniaOVERn2.png}
    \caption{Wykres wartości $L_n^{(2)}$}
    \label{fig:Ln1}
\end{figure}
\begin{figure}[ht]
    \centering
    \includegraphics[width=0.8\textwidth]{wykresy/porownaniaOVERn.png}
    \caption{Wykres wartości $\frac{cmp(n)}{n}$}
    \label{fig:cmpOvern}
\end{figure}
\begin{figure}[ht]
    \centering
    \includegraphics[width=0.8\textwidth]{wykresy/porownaniaOVERn2.png}
    \caption{Wykres wartości $\frac{cmp(n)}{n^2}$}
    \label{fig:cmpOvern2}
\end{figure}
\end{document}
